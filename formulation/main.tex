\documentclass{article}
\usepackage{graphicx} % Required for inserting images
\usepackage{fullpage}
\title{Quantum Adiabatic Computation}
\author{Hrishikesh Belagali}
\usepackage{pdflscape}
\date{October 2024}
\usepackage{physics}
\usepackage{amssymb}
\begin{document} 
\maketitle
\section{Introduction}
Quantum Adiabatic Computation utilizes the adiabatic theorem of quantum 
mechanics which states that a time-dependent Hamiltonian in an eigenstate will remain 
in that eigenstate when subject to evolution, given that the evolution is adiabatic in nature. 
In this context, adiabatic refers to an evolution rate much slower than the intrinsic frequency of the 
quantum system. \\
This is significant for optimization problems. One can start with a simple Hamiltonian with a trivial groundstate, such as a quantum Ising chain without a transverse field. 
The objective function must be framed in the form of a Hamiltonian. A time-dependent Hamiltonian can be created by interpolating the initial and final Hamiltonians smoothly. 
Then, if the time-dependent Hamiltonian is evolved, the ground state of the final Hamiltonian can be determined. 
\subsection{Limitations}
Quantum adiabatic algorithms scale rather poorly because Pauli matrices for N particles are order $2^N$ matrices. There must be reasonable gaps between the energy values of various eigenstates. 
Hence, this method does not work for degenerate systems. The method also fails if there are level crossings in the time evolution of the eigenstate. 
\section{The Problem}
This project will optimize a transverse-field Ising chain with a transverse field. The Hamiltonian is given by 
$$\hat H_p = -J\pqty{\sum_{i}Z_iZ_{i+1} + g\sum_{i}X_i}$$
\subsection{The Initial Hamiltonian}
As mentioned, the initial Hamiltonian is constructed to have a trivial ground state. This can be a transverse-field Hamiltonian without a transverse field. 
$$\hat H_i = -J\sum_{i}{Z_iZ_{i+1}}$$ 
The trivial solutions are as follows: 
\subsubsection{The ferromagnetic case: $J>0$}
The particles align themselves in the direction as $\ket{\cdots \uparrow\uparrow\uparrow \cdots}$ or $\ket{\cdots \downarrow\downarrow\downarrow \cdots}$

\subsubsection{The antiferromagnetic case: $J<0$}
The particles align themselves in an alternate fashion- $\ket{\cdots \uparrow\downarrow\uparrow \cdots}$
 
\subsection{The Interpolated Hamiltonian}
We create a time parameter $S: \bqty{0, T_f}\to \bqty{0,1}$. The use of this parameter is akin to gradually increasing a transverse field. We choose the interpolation function 
The interpolation chosen for this implementation $$S(t) = \frac{t}{T_f}$$
$$\hat H(t) = -J \pqty{\sum_{i}Z_{i}Z_{i+1} + S(t)g\sum_i X_i} $$
Note that $\hat H(0) = \hat H_i$ and $\hat H(t_f) = \hat H_p$. Starting with the appropriate ground state depending on the sign of $J$, one can adiabatically evolve to the final Hamiltonian and model the quantum phase transition. 
 
\end{document}
