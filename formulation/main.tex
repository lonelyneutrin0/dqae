\documentclass{article}
\usepackage{graphicx} % Required for inserting images
\usepackage{fullpage}
\title{Discretized Quantum Adiabatic Computation}
\author{Hrishikesh Belagali}
\usepackage{pdflscape}
\date{October 2024}
\usepackage{physics}
\usepackage{amssymb}
\begin{document} 
\maketitle
\section{Introduction}
This project leverages the Trotter-Suzuki Approximation to discretize Hamiltonian time evolution. By doing so, it proposes a classical algorithm that simulations quantum adiabatic computation. 
QUBO problems, which are formulated as Ising models are solved through the adiabatic evolution of a Hamiltonian with a trivial groundstate. 
\subsection{Limitations}
Quantum adiabatic algorithms scale rather poorly because Pauli matrices for N particles are order $2^N$ matrices. There must be reasonable gaps between the energy values of various eigenstates. 
Time interpolation must be carefully controlled to prevent eigenvalue quasi-crossings. 

\section{Algorithm}
\subsection{Notation}
The standard computational basis is used for this implementation, which is as follows: 
$\ket 0 = \begin{bmatrix}
    1 \\ 0
\end{bmatrix}$ and $\ket 1 = \begin{bmatrix}
    0 \\ 1
\end{bmatrix}$. 
The corresponding eigenvalues subject to the Pauli-Z quantum gate yield the eigenvalues +1 and -1 respectively- 
$$\sigma_z \ket 0 = \ket 0$$ 
$$\sigma_z \ket 1 = -\ket 1$$ 
The application of a Hadamard gate transforms the qubits to a Hadamard basis- 
$$\mathcal H \ket 0 = \ket + = \frac{1}{\sqrt 2} \pqty{\ket 0 + \ket 1}$$
$$\mathcal H \ket 1 = \ket - = \frac{1}{\sqrt 2} \pqty{\ket 0 - \ket 1}$$


\subsection{The Problem}
\begin{align}
    f(\mathbf x) &= \mathbf x^T Q \mathbf x = \sum_{i=1}^n\sum_{j=1}^n{Q_{ij}x_ix_j} \\ 
    \intertext{Let $\sigma_i = 2\mathbf x_i - 1 \implies \mathbf x_i = \frac{\sigma_i + 1}{2}$ }  \\ 
    f(\mathbf \sigma) &= \sum_{i=1}^n \sum_{j=1}^n Q_{ij}\pqty{\frac{\sigma_i + 1}{2}}\pqty{\frac{\sigma_j + 1}{2}} \\ 
    &= \frac14\sum_{i=1}^n\sum_{j=1}^n Q_{ij}(\sigma_i\sigma_j + \sigma_i + \sigma_j + 1) \\ 
    \intertext{The constant can be neglected throughout the optimization process, but must be added at the end to ensure an accurate result.}
    &= \frac14 \sum_{ij}Q_{ij}\sigma_i\sigma_j + \frac14 \sum_i (Q_{ij}+Q_{ji})\sigma_i + \frac14\sum_{ij} Q_{ij} \\ 
    &= -\sum_{i} {J_{ij}\sigma_i\sigma_j} + \sum_{i}h_i \sigma_i + C 
\end{align}
 where $$J = -\frac{Q}{4}$$ $$h_i = \frac14 \sum_i {Q_{ij} + Q_{ji}}$$ $$C = \frac14 \sum_{ij}{Q_{ij}}$$

Note that for an N-order QUBO, N qubits are required, which results in a system size of $2^N$.
\subsection{The initial Hamiltonian}
The initial Hamiltonian must not commute with the final Hamiltonian to ensure a proper evolution. It must also have a trivial groundstate. For this implementation, 
a tranverse field Hamiltonian is chosen- 
$$H = -\sum_{i}\sigma_i^X$$
The groundstate eigenvector is $\frac{1}{\sqrt 2}\pqty{\ket 0 + \ket 1}^{\otimes N}$. 
$$H \ket +^{\otimes N} = -N$$
The groundstate can be prepared easily using the Hadamard gate- 
$$\mathcal H ^{\otimes N}\ket 0^{\otimes N} = \ket + ^{\otimes N}$$

\subsection{The final Hamiltonian}
The final Hamiltonian is the quantum version of the Ising model represented above. 
$$H = - \sum_{ij} J_{ij}\sigma_i^Z\sigma_j^Z + \sum_i h_i \sigma_i^X$$
\subsection{The adiabatic Hamiltonian}
$$H(t) =  (1-s(t))H_i + s(t)H_f$$
where $s : [t, T] \mapsto [0,1]$ 

\subsection{The Gap Function}
$$g(s) =  E_1(s) - E_0(s)$$
It defines the lowest excitation energy at each step of the evolution. It is significant because the total evolution time has the condition $$T >> \frac{1}{\text{min} \{g(s)\} }$$

\subsection{Time Evolution}
The quantum state and groundstate are evolved adiabatically through the discretized time evolution operator. For small timesteps, the time evolution operator is defined as 
\begin{align}
    U(t, t + \Delta t) &\approx \exp{-i\Delta t H(t)} \\ 
    U(t, t+\Delta t) &\approx \exp{-i \frac{\Delta t}{2}\frac{t}{T}H_f}\exp{-i \Delta t\pqty{1-\frac{t}{T}} \frac{t}{T} H_i}\exp{-i \frac{\Delta}{2}\frac{t}{T}H_f}\\ 
    \intertext{The above equation is a second order Trotter decomposition. The quantum state is evolved as shown below.}
    \ket \psi_{t + \Delta t} &= U(t, t+\Delta t) \ket \psi
\end{align}


\end{document}
